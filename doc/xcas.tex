% !TEX TS-program = pdflatex
% !TEX encoding = UTF-8 Unicode

\documentclass[12pt]{article} % use larger type; default would be 10pt

\usepackage[utf8]{inputenc} % set input encoding (not needed with XeLaTeX)

\usepackage{geometry} % to change the page dimensions
\geometry{letterpaper} % or letterpaper (US) or a5paper or....
\usepackage{graphicx} % support the \includegraphics command and options

%%% PACKAGES
\usepackage{booktabs} % for much better looking tables
\usepackage{array} % for better arrays (eg matrices) in maths
\usepackage{paralist} % very flexible & customisable lists (eg. enumerate/itemize, etc.)
\usepackage{verbatim} % adds environment for commenting out blocks of text & for better verbatim
\usepackage{subfig} % make it possible to include more than one captioned figure/table in a single float

%%% HEADERS & FOOTERS
\usepackage{fancyhdr} % This should be set AFTER setting up the page geometry
\pagestyle{fancy} % options: empty , plain , fancy
\renewcommand{\headrulewidth}{0pt} % customise the layout...
\lhead{}\chead{}\rhead{}
\lfoot{}\cfoot{\thepage}\rfoot{}

%%% SECTION TITLE APPEARANCE
\usepackage{sectsty}
\allsectionsfont{\sffamily\mdseries\upshape} % (See the fntguide.pdf for font help)
% (This matches ConTeXt defaults)

%%% ToC (table of contents) APPEARANCE
\usepackage[nottoc,notlof,notlot]{tocbibind} % Put the bibliography in the ToC
\usepackage[titles,subfigure]{tocloft} % Alter the style of the Table of Contents
\renewcommand{\cftsecfont}{\rmfamily\mdseries\upshape}
\renewcommand{\cftsecpagefont}{\rmfamily\mdseries\upshape} % No bold!

%%% END Article customizations

%%% The "real" document content comes below...

\title{XCas}
\author{Scott Cyphers}

\begin{document}
\maketitle
\abstract
XCas is a simple design pattern that augments the Apache UIMA CAS to provide modular type-safe access to external resources.  Every potential user
of an XCas defines the interface that it requires.  A pipeline defines one or more XCas resources that provide access to implementations of the required interfaces.  At appropriate points during processing, a CAS is bound to such an implementation.  Annotators include the resource specification and
obtain the XCas implementation from their CAS during processing.

\section*{Annotator Definition}
Annotators define a public static \texttt{XCas} interface with all their required methods.  The class \texttt{XCasAnnotator\_ImplBase<XCas>}
provides a convenient mechanism for declaring the external resource.  For example, Figure~\ref{FIG:SimpleAnnotator} defines an annotator
that require one XCas method.
\begin{figure}
\begin{verbatim}
public class SignalWordsAnnotator
extends XCasAnnotator_ImplBase<SignalWordsAnnotator.XCas> 
{
   public static interface XCas
   {
      SignalWords getSignalWords();
   }
	
   public static <T extends XCas> 
   AnalysisEngineDescription getDescription(XCasResource<T> resource)
   throws ResourceInitializationException 
   {
      return getDescription(SignalWordsAnnotator.class, resource);
   }

   @Override
   public void process(JCas jCas, XCas xCas) 
   throws AnalysisEngineProcessException 
   {
        xCas.getSignalWords().compute(jCas);
   }
}
\end{verbatim}
\caption{Simple annotator definition}
\label{FIG:SimpleAnnotator}
\end{figure}
The \texttt{getDescription} method returns a description for this annotator, given a resource that implements the interface.  The
\texttt{XCasAnnotator\_ImplBase} implements the \texttt{process(JCas jCas)} method, obtains the associated XCas, and passes
both to \texttt{process}.
\section*{Resource Definition}
Resource definition is also simple.  The resource defines an public static \texttt{XCas} interface that extends a set of \texttt{XCas} interfaces
which includes the \texttt{XCas} interfaces for the annotators that will use it.
\begin{figure}
\begin{verbatim}
public class QuestionASResource 
extends XCasResource_Impl<QuestionASResource.XCas> 
{
   public static interface XCas extends
   ChunkVectorAnnotator.XCas,
   ClosestChunkAnnotator.QuestionXCas,
   ClosestSentenceAnnotator.QuestionXCas,
   CommentInitializer.QuestionXCas,
   Configuration.QuestionXCas,
   Doc2VecAnnotator.XCas,
   Doc2VecFeatureVectorAnnotator.QuestionXCas,
   DocumentInitializer.XCas,
   DocumentVectorAnnotator.XCas,
   QuestionCommentCASMultiplier.XCas,
   SentenceVectorAnnotator.XCas,
   TokenVectorAnnotator.XCas,
   TreeStringAnnotator.XCas,
   WordWeightAnnotator.XCas
   {}

   public final static
   QuestionASResource resource = new QuestionASResource();
	
   @Override
   public Object getResource() 
   {
      return resource;
   }
}
\end{verbatim}
\caption{\texttt{XCas} Resource definition}
\label{FIG:ResourceDefinition}
\end{figure}
The resource is a singleton that associates a \texttt{JCas} with an implementation of its \texttt{XCas} interface.  The class
\texttt{XCasResource\_Impl<XCas>} extends  \texttt{ExternalResourceLocator} to provide methods for managing the associations, as well as a \texttt{getResourceDescription} method
which creates a \texttt{ExternalResourceDescription} for this resource.
\section*{\texttt{XCas} Implementation}
Any class that implements the the resource's \texttt{XCas} interface and be bound to a \texttt{JCas} by the resource.  For example, Figure~\ref{FIG:XCasImplementation} shows an implementation of \texttt{QuationASResource.XCas}.
\begin{figure}
\begin{verbatim}
public class QuestionAS
implements QuestionASResource.XCas
{
   @Override
   SignalWords getSignalWords()
   {
      ...
   }

   ...
}
\end{verbatim}
\caption{Implementation of \texttt{XCas}}
\label{FIG:XCasImplementation}
\end{figure}
\section*{Setting up a Pipeline}
\section*{Binding an \texttt{XCas} to a \texttt{JCas}}
\section*{XCas Strategies}
\section*{Implementation}
\end{document}